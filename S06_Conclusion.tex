% Section:Conclusion

\subsection{Final Remarks}
The NPAG program was developed at the USC Laboratory of Applied Pharmacokinetics. James Burke (University of Washington) developed the Primal-Dual Interior-Point  method discussed in the Appendix.
%
Robert Leary (Pharsight Corporation) developed the Adaptive Grid method and  wrote the original Fortran program for NPAG. 
%
Michael Neely, MD (USC Children's Hospital of Los Angeles) developed the program package Pmetrics which contains NPAG as a subprogram. Pmetrics  is an R package for nonparametric and parametric population modeling and simulation and is available at  {\tt www.lapk.org}, see \citet{Neely2011PMetrics}.
%
\subsection{Conclusions}
We have desribed a nonparametric maximum likelihood method called NPAG for estimating multivariate mixing distributions. NPAG is based on an iterative algorithm employing
the Primal-Dual Interior-Point method and an Adaptive Grid method.
Our method is able to handle high-dimensional and complex mixture models.  Other methods are discussed. A detailed description of NPAG is given. The important application to population pharmacokinetics is described and a non-trivial example is given. 

In addition to population pharmacokinetics, this research also applies to empirical Bayes estimation, see
\citet{KoenkerMizera14} and to many other areas of applied mathematics, see \citet{Banks12}.





	